\documentclass{article}
\usepackage{hyperref}

\begin{document}

\section{Events. Traces}

\subsection{Events}

A mask is a map from signal\_numbers to enabledness values.

An event is caracterized by the following attributes:

\begin{itemize}
\item instance\_id --- identifying the current thread/signal instance  
\item pc --- program counter capturing the execution order within source
\item type of event: 
\begin{itemize}
\item START --- event starting a thread/signal
\item END 
\item WRITE
\item READ
\item ATOMIC\_WRITE
\item ATOMIC\_READ
\item LOCK
\item UNLOCK
\item WRITE\_MASK
\item READ\_MASK
\item ESTABLISH\_SIGNAL signal\_number, source\_id, DEFAULT\_MASK 
\end{itemize}
\item extra attribute for START
\begin{itemize}
  \item handler\_id --- identifier for the handler treating this signal
\end{itemize}
\item extra attributes for WRITE/READ/ATOMIC\_WRITE/ATOMIC\_READ
\begin{itemize}
\item location --- location being accessed
\item value --- value written/read
\end{itemize}
\item extra attribute for LOCK/UNLOCK
\begin{itemize}
\item lock\_id --- identifier for the resource being locked/unlocked
\end{itemize}
\item extra attribute for WRITE\_MASK/READ\_MASK
\begin{itemize}
\item mask --- the mask being read / written
\end{itemize}
\item extra attribute for ESTABLISH\_SIGNAL
\begin{itemize}
  \item signal\_number 
  \item handler\_id --- identifier for the handler treating this signal
  \item default\_mask --- the default mask to start this signal with
\end{itemize}
\end{itemize}

\section{Concurrent object, serial specification}

We adopt the definition of concurrent objects and serial specifications proposed
by Herlihy and Wing \cite{herlihy-wing-1990}.

A concurrent object is behaviorally defined through a set of atomic operations,
which any thread can perform on it, and a serial specification of its legal
behavior in isolation.

The serial specification describes the valid sequences of operations which can
be performed on the object.


\subsection{Sequential execution order}

Is defined for any piece of code which is supposed to be executed sequentially 
(the fact that it can be interrupted will be modelled elsewhere).

Given two events $e_1$ and $e_2$, we say that $e_1\prec e_2$

\begin{thebibliography}{00}

\bibitem{herlihy-wing-1990} Maurice P. Herlihy and Jeannette M. Wing. 1990.
Linearizability: a correctness condition for concurrent objects.
ACM Trans. Program. Lang. Syst. 12, 3 (July 1990), 463-492.
DOI=\href{http://dx.doi.org/10.1145/78969.78972}{10.1145/78969.78972}
\end{thebibliography}

\end{document}